\section{Использование POST запроса с использованием Proxy}

Интернет --- всемирная система объединенных компьютерных сетей, построенная на базе IP и маршрутизации IP-пакетов. Интернет образует глобальное информационное пространство, служит физической основой для Всемирной паутины, изначально создававшейся как глобальная библиотека, позволяющая только передавать клиентам документы в электронном виде. Однако этого было не достаточно. Интернет-ресурсам требовался обмен информации, при котором клиент мог отправить данные на HTTP сервер. Для этой цели был разработан протокол Get. Данный стандарт сильно ускорил развитие интернета, однако он имеет некоторые недостатки, которые исправил протокол Post.

Поскольку, используя данные средства можно передавать данные на удаленный сервер, то возникло предположение, что подобным способом можно организовать скрытую перечу информации незаметно для DLP системы.

Реализация данного метода оказалось примитивной. Потребовалось организовать Http сервер и реализовать программу, передающую данные через Get/Post. 

Тестирование данного метода было успешным. Однако в каждой DLP системе встроен firewall, способный детектировать соединения недостоверных программ к ненадежным ресурсам. В данном случае COMODO не справился с данной задачей и позволил передать данные. Но в общем случае данный способ не должен работать. Кроме того в каждый firewall встроен список вредоносных ресурсов, который он постоянно обновляет. Если ресурс попадет в этот список, то система защиты не позволит подключиться к данному серверу. Поэтому данный канал является ненадежным. Было предположено, что для обхода данной защиты можно воспользоваться Proxy-сервером, который возьмет на себя организацию подключения к вредоносному ресурсу.


Прокси-серверы появились на заре эпохи Интернета, когда пользователей этой сети становилось все больше и больше, а внешние IP-адреса стоили немалых денег. Тогда основным назначением proxy-серверов являлась организация доступа в Интернет локальных пользователей без добавления их компьютеров к Глобальной сети, то есть без назначения внешних IP-адресов компьютерам, а выход в Интернет осуществлялся только с одного внешнего IP-адреса. Слово proxy в переводе с английского означает «доверенное лицо» или «представитель». Условно говоря, прокси-сервер действует от лица клиента в Интернете, и для других пользователей Сети виден только сам сервер, а не клиент. Таким образом, кроме общего доступа в Интернет локальных пользователей, которые не имеют прямого выхода в Сеть, такие серверы позволяют соблюсти приватность работы в Интернете. Вследствие того, что компьютеры обычных пользователей не размещены непосредственно в Сети, снижается угроза хакерских атак, поскольку прямого доступа к компьютерам локальной сети нет.

Существует несколько типов прокси-серверов, каждый из которых имеет узкую специализацию, то есть поддерживает работу только с одним или несколькими протоколами. Самыми распространенными на данный момент являются http-, Socks- и NAT-прокси. Последние входят в стандартные компоненты современных операционных систем, таких как Linux и Windows. По своим характеристикам программные прокси-серверы NAT практически не отличаются от аппаратных (маршрутизаторов) и существенно уступают в администрировании узкоспециализированным прокси-серверам. 

Организовав Proxy-сервер на базе 3Proxy, была произведена незначительная модификация программы и проведено повторное тестирование, которое показало, что DLP не смогло предотвратить соединение данной программы к прокси-серверу, что позволило передать данные. COMODO в данном случае также не смог предотвратить недостоверное соединение, передающее защищаемые данные.