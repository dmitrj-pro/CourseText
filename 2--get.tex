\subsection{HTTP Get и Post - запросы}

HTTP (HyperText Transfer Prоtocоl — "протокол передачи гипертекста"). 

В компьютерной терминологии, гипертекст — текст, сформированный с помощью языка разметки, потенциально содержащий в себе гиперссылки, то есть ссылки на другой элемент в самом документе, а также на другой объект, расположенный на локальном диске или в компьютерной сети, либо на элементы этого обьекта.

Основой HTTP является технология «клиент-сервер», то есть предполагается существование потребителей (клиентов), которые инициируют соединение и посылают запрос, и поставщиков (серверов), которые ожидают соединения для получения запроса, производят необходимые действия и возвращают обратно сообщение с результатом.

Данный протокол в последнее время получил широкое применение для обмена информацией между клиентом и сервером. С помощью POST, GET и HEAD HTTP запросов пользователь может предать необходимые данные на сервер. Данный протокол является простым в использовании и реализации, что его делает популярным каналом для передачи информации на удаленный сервер, используемым в вредоносных программах, самыми извеcnными из которых являются Carberp (POST, GET), SpyEye (POST), Storm, ZeuS (POST, GET).

Отличительным средством данного протокола является скрытие настоящего сервера, принимающего и обрабатывающего информацию. Данная функция основывается на использовании Http-Proxy сервера, который подменяет собой скрываемый сервер и сам выполняет запрос к нужному компьютеру, передавая данные от клиента. Данная функция была введена в глобальную сеть из-за просчетов при создании стандарта TCP/IP. Данный стандарт создавался во время малочисленных компьютеров, которые в основном принадлежали крупным университетам, и предполагал, что у каждого компьютера будет свой адрес, однозначно идентифицирующий его. Но прогресс не стоит на месте. В скором времени таких адресов стало резко не хватать (что происходит и по сей день). Самым простым решением данной проблемы стало введение Proxy-серверов, которые в последующее время незначительно изменились и получили название NAT (<<транслятор сетевых адресов>>). Но оригинальные Proxy сервера существуют и по сей день. Ими и пользуются злоумышленники для своих программ (например Wirenet).