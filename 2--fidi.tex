\section{Виды утечек}

Утечка информации возможна по разным причинам, а именно:

\begin{description}
	\item[-]
		Умышленные утечки информации – это случаи преднамеренной утечки данных, когда пользователь, имеющий доступ к ценной информации, знал о возможных негативных последствиях своих действий, понимал, что такие действия носят противоправный характер. Кроме того, сотрудник организации получал предупреждение об ответственности, но все равно передавал данные или создавал условия для их утечки, желая получить материальное вознаграждение или иную выгоду для себя. В результате действий сотрудника возникли условия, способствующие потере контроля над информацией, нарушению конфиденциальности данных. При этом не имеет значения, были ли негативные последствия для компании или отдельных лиц от действий, совершенных инсайдером;

	\item[-]
		Кража информации (извне). Взлом компьютера с помощью вредоносных программ и похищение информации с целью использования в корыстных интересах;  
	\item[-]
		Взлом программного обеспечения. На хакерские атаки приходится 15\% от всей утечки информации. Вторжение в устройство извне и незаметная установка вредоносных программ позволяет хакерам полностью контролировать систему и получать доступ к закрытым сведениям, вплоть до паролей к банковским счетам и картам. Для вторжения извне в устройство могут применяться различные программы типа трояна. Главное отличие этого вида утечки – активные действия внешних лиц с целью доступа к информации;
	\item[-] 
	    Кражи носителей. Достаточно распространенный способ утраты случается в результате преднамеренной кражи устройств с информацией. В большинстве случаев это случается из-за кражи ноутбуков, смартфонов, планшетов и других съемных носителей данных в виде флэшек, жестких дисков;
	\item[-] 
	    Случайные утечки. К этому виду можно отнести веб-утечки, которые чаще всего происходят в силу неосведомленности или ошибочных действий сотрудников организации. Такой вид утраты случается в результате размещения конфиденциальной информации в интернет. Также, не последнюю роль играет человеческий фактор, когда сотрудник умышленно или без умысла позволяет получить доступ к закрытым данным всем желающим.
\end{description}

Рассмотр утечек, зависящие от человеческого фактора можно опустить, поскольку данные способы в большем случае не подвергаются контролю DLP систем и их почти невозможно контролировать. Данные системы в большей части отслеживают все каналы, через которые возможно передавать информацию. Подобным функционалом обладает Firewall, который лежит в основе каждой DLP системы и обладающий дополнительными возможностями.

В настоящий момент существует множество протоколов, через которые можно передавать данные. Данное условие сильно усложняет работу межсетевым экранам.



