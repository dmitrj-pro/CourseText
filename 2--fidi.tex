\section{Виды утечек}

Утечка информации возможна можно по разным причинам, а именно:

\begin{itemize}
	\item
		Умышленные утечки 
	\item
		Инсайдеры. К этому виду потери данных относятся случаи, основной причиной которой стали действия сотрудников, имеющих доступ к секретам легально, в силу своих служебных обязанностей. Все случаи инсайда условно можно разделить на две группы, в одном случае сотрудник, не имея доступа к информации, сумел незаконно ее получить, а в другом при официальном доступе к закрытым данным сотрудник умышленно вынес ее за пределы компании; 
	\item
		Кража информации (извне). Взлом компьютера с помощью вредоносных программ и похищение информации с целью использования в корыстных интересах;  
	\item
		Взлом программного обеспечения. На хакерские атаки приходится 15\% от всей утечки информации. Вторжение в устройство извне и незаметная установка вредоносных программ позволяет хакерам полностью контролировать систему и получать доступ к закрытым сведениям, вплоть до паролей к банковским счетам и картам. Для вторжения извне в устройство могут применяться различные программы типа трояна. Главное отличие этого вида утечки – активные действия внешних лиц с целью доступа к информации; 
	\item 
	    Вредоносные программы (бекдоры, трояны) нацелены на причинение вреда владельцу устройства, позволяют незаметно проникать в систему, похищать информацию с помощью копирования, искажать или полностью удалять ее, или подменять другой похожей информацией. 
	\item 
	    Избыточные права
	\item 
	    Кражи носителей    . Достаточно распространенный способ утраты случается в результате преднамеренной кражи устройств с информацией. В большинстве случаев это случается из-за кражи ноутбуков, смартфонов, планшетов и других съемных носителей данных в виде флэшек, жестких дисков;
	\item 
	    Случайные утечки. К этому виду можно отнести веб-утечки, которые чаще всего происходят в силу неосведомленности или ошибочных действий сотрудников организации. Такой вид утраты случается в результате размещения конфиденциальной информации в интернет. Также, не последнюю роль играет человеческий фактор, когда сотрудник умышленно или без умысла позволяет получить доступ к закрытым данным всем желающим.
\end{itemize}

       



  










