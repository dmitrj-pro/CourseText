\section{Использование почтового сервера}

На данный момент электронная почта очень активно используется в бизнес-среде, так как она представляет собой уникальную среду общения, которая может использоваться в качестве оперативного и очень недорогого средства для связи как между сотрудниками компании, так и с партнерами и клиентами. Правда, для того, чтобы ее использование получило максимальную эффективность, компании необходимо организовать собственный почтовый сервер. Только в этом случае корпоративная почта становится легко управляемым, безопасным, а также наиболее удобным и дешевым в применении бизнес-инструментом.

Подобная информация натолкнула на еще один способ организации скрытого канала, основанного на том, что вредоносная программа может заполучить данные для доступа к данному почтовому серверу и воспользоваться им для передачи информации, прикрываясь одним из сотрудников, чьи данные удалось заполучить программе.

При анализе данного способа построения скрытого канала не учитывалась возможность получения данных клиента почтового сервера. Предполагалось, что злоумышленник может просканировать приложения, установленные на компьютере, выделить среди них почтовые приложения и получить данные авторизации. Поэтому в ходе эксперимента использовался локальный SMTP сервер, через который передавались данные. 

Таким образом, при реализации данного канала был зарегистрирован тестовый адрес электронной почты, на который производилась отправка данных, была реализована программа, подключающаяся к SMTP серверу и передающая через него электронные письма, был запущен Softstack SMTP сервер на тестовом компьютере.

В ходе эксперимента COMODO никаким образом не предотвратил передачу информации через локальный почтовый сервер и установление скрытого канала передачи данных было успешным. Однако следует учесть, что некоторые DLP системы не позволяют передавать электронные письма без разрешения на то администратора или подробно не изучив содержимое электронного письма..



