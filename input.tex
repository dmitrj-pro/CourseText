\addcontentsline{toc}{chapter}{Введение}
\chapter*{Введение}

На последний момент возрастает актуальность темы защиты данных от утечек.
 
Говоря об информационной безопасности, в настоящее время имеют в виду безопасность компьютерную. Действительно, информация, находящаяся на электронных носителях играет все большую роль в жизни современного общества. Уязвимость такой информации обусловлена целым рядом факторов: огромные объемы, многоточечность и возможная анонимность доступа, возможность "информационных диверсий". Все это делает задачу обеспечения защищенности информации, размещенной в компьютерной среде, гораздо более сложной проблемой, чем, скажем, сохранение тайны традиционной почтовой переписки.

Если говорить о безопасности информации, сохраняющейся на традиционных носителях (бумага, фотоотпечатки и т.п.), то ее сохранность достигается соблюдением мер физической защиты (т.е. защиты от несанкционированного проникновения в зону хранения носителей). Другие аспекты защиты такой информации связаны со стихийными бедствиями и техногенными катастрофами. Таким образом, понятие компьютерной информационной безопасности в целом является более широким по сравнению с информационной безопасностью относительно традиционных носителей.

Если говорить о различиях в подходах к решению проблемы информационной безопасности на различных уровнях (государственном, региональном, уровне одной организации), то такие различия просто не существуют. Подход к обеспечению безопасности Государственной автоматизированной системы "Выборы" не отличается от подхода к обеспечению безопасности локальной сети в маленькой фирме.

Потому жизненно необходимы методы защиты информации для любого человека, использующего компьютер. По этой причине практически любой пользователь ПК в мире так или иначе «подкован» в вопросах борьбы с вирусами, «троянскими конями» и другими вредоносными программами, а также личностями стоящими за их созданием и распространением — взломщиками, спамерами, крэкерами, вирусмэйкерами (создателями вирусов) и просто мошенниками, обманывающих людей в поисках наживы — корпоративной информации, стоящей н
емалых денег.

Для решения данной проблемы были реализованы средства предотвращения утечек информации, реализующих систему защиты данных от несанкционированного доступа к ней. Но злоумышленники придумывают новые способы заполучения информации. Поэтому важно проверять системы защиты на возможность кражи информации новыми методами

Цель: \textit{Поиск недостатков существующих методов выявления скрытых каналов из защищаемых сетей}
	
Задачи:

\begin{enumerate}
	\item
		Исследование и анализ известных DLP систем;
	\item
		Разработать и реализовать методы обхода средств контроля исходящего сетевого трафика;
	\item
		Развернут экспериментальный стенд;
	\item
		Провести экспериментальные исследования, подтверждающие предположения о неэффективности современных средств защиты информации.
\end{enumerate}

	