\subsection{DNS - Domain Name System}

Каждый компьютер в Интернете имеет свой IP-адрес – 4 числа от 0 до 255. Такой адрес удобен при маршрутизации, так как определяет месторасположение компьютера в сети Интернет. Однако, такие числа неудобны для восприятия пользователем и вызывают проблемы при смене IP адреса у компьютера.

В сети ARPANET соответствие между текстовыми и двоичными адресами записывалось в файле host.txt, в котором перечислялись все хосты и их IР-адреса. Каждую ночь все хосты получали этот файл с сервера, на котором он хранился. В сети, состоящей из нескольких сотен устройств, работающих под управлением системы с разделением времени, такой подход работал вполне приемлемо. 

На смену «однофайловой» схеме пришел DNS, являющийся иерархической схемой имен, основанной на доменах, и распределенной базе данных. В первую очередь эта система используется для преобразования имен хостов и пунктов назначения электронной почты в IР-адреса, но также может использоваться и в других целях.

Данный протокол на данный момент только набирает популярность среди разработчиков вредоносных программ. С помощью него функционируют вредоносные программные обеспечения, такие как ProjectSauron и Multigrain. 