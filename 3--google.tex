\section{Использование средств Google}

Интернет со времени его возникновения сильно изменился. Он превратился из службы доставки документов в сервис доставки веб-приложений, предоставляющих различный функционал для пользователя будь то обмен сообщений между пользователями или вычисление сложных математических выражений. Интернет превратился в распределенную систему, каждое звено которой выполняет свою роль. Среди всех этих ресурсов можно выделить легетимные сервисы, облегчающие выполнение поставленной задачи. Зачастую такими сервисами пользуются злоумышленники, поскольку такие ресурсы зачастую сетевые экраны не проверяют, а у системного администратора, отвечающего за безопасность сети, обращение к ним не вызовет интереса. 

Одним из таких ресурсов является Google Forms, позволяющий создавать публичные опросы и анкеты. Данный сервис получил огромную популярность среди тех, кто проводить сбор данных. 

Возникло предположение, что воспользовавшись данным сервисов не по на значению, можно передавать данные от зараженного компьютера к злоумышленнику в обход DLP систем, тем самым организовав скрытый канал передачи данных.

Для реализации данного метода потребовалось зарегистрировать в системе Google Forms анкету с полями для приема данных. Проанализировав исходный код полученного опросника, выделил из него параметры POST запроса, использующие для передачи информации системе. Далее реализовал программу, обращающуюся к Google Forms и передающую ей POST запрос, содержащий защищаемую информацию, защищенную шифрованием методом сдвига.

На практике COMODO сразу заметил, что недостоверная программа обращается к легитимному сайту, защищенному SSL сертификатом, о чем DLP доложила администратору сети. Если опустить данный факт из внимания, то данный канал является работоспособным и подобным образом можно реализовать канал передачи данных, который DLP системы могут не обнаружить.

