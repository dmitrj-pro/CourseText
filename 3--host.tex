\section{Подмена host файла}

Когда глобальная сеть Интернет только зарождалась, для установления связи между компьютерами использовались IP адреса. Каждый клиент глобальной сети должен был знать IP адрес компьютера, с которым он хотел установить соединение для передачи данных. Данная система была сильно неудобная, поскольку нужно было хранить адреса удаленных компьютеров и при смене компьютера заменялся его адрес, что давало дополнительные расходы. Для решения данной проблемы решили использовать доменные имен (символьное имя, служащее для идентификации областей — единиц административной автономии в сети Интернет — в составе вышестоящей по иерархии такой области). 

Самым простым способом организации доменов являлся файл host, позволяющий сопоставлять имена с IP адресами компьютеров на уровне операционной системы. Данный способ также был неудобен для использования обычными пользователями сети, но он дал основу для организации более совершенных систем, включая DNS. 

Не смотря на свой возраст данный файл до сих пор используется в каждой операционной системе для организации технического DNS, используемого для отладки оборудования и программных средств.

Воспользовавшись данной информацией, была выдвинута гипотеза, что исправить недостаток предыдущего метода (использование POST запроса с Proxy) можно, модифицировав данный файл, добавив в него домен легетимного сайта, указывающего на сервер, принимающий информацию от нашей программы, знающей об этой модификации.

Реализация данного метода потребовала незначительной модификации программы из предыдущего метода передачи данных, добавив в нее редактирование файла host. 

При проведении эксперимента COMODO Firewall сразу заметил попытку изменения системного файла host и заблокировал ее. При дальнейшем анализе было выяснено, что данная DLP система заставляет игнорировать операционную систему данный файл, что не позволяет организовать канал передачи данных таким образом.
