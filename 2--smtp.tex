\subsection{Почтовые сервера}

В 1960-х годах использовались различные виды электронной связи. Люди связывались друг с другом с помощью систем, разработанных для мейнфреймов. Когда всё больше компьютеров становились связанными ARPANET, были разработаны стандарты для того, чтобы пользователи на различных системах могли писать электронные сообщения друг другу. Эти стандарты, разработанные в 1970-х годах, стали основой для SMTP.

Начало почтовых серверов можно проследить в двух описанных в 1971 г. реализациях — Mail Box Protocol и SNDMSG, который был изобретён Рэем Томлинсоном из BBN Technologies для TOPS-20/TENEX-компьютеров, посылающих сообщения по ARPANET.

Дальнейшие реализации включают в себя FTP Mail и Mail Protocol, разработанные в 1973 г. Разработка продолжалась на протяжении 1970-х, пока ARPANET не преобразовалась в современный Интернет около 1980 г. В том же году Джон Постел предложил Mail Transport Protocol (протокол передачи почты), благодаря которому FTP перестал быть основой для передачи почты. SMTP опубликован в RFC 821 в августе 1982 г.

SMTP стал широко использоваться в ранние 1980-е. В то время он был дополнением для работающей под Unix почтовой программы Unix Copy Program (UUCP), которая больше подходила для обработки передачи электронных сообщений между периодически связанными устройствами. С другой стороны, SMTP прекрасно работает, когда как отправляющее, так и принимающее устройства связаны в сети постоянно. Оба устройства используют механизм хранения и пересылки и являются примером push-технологии.

Предоставление сообщений (RFC 2476) и SMTP-AUTH (RFC 2554) были введены в 1998 и 1999 гг. и описывали новые тенденции в передаче электронных сообщений. Изначально, SMTP-сервера были обычно внутренними для организации, получая сообщения от организаций извне и ретранслируя сообщения организации во внешнюю среду. Но с течением времени, SMTP-сервера расширяли свои функции и в стали агентами предоставления сообщений для пользовательских почтовых приложений, некоторые из которых теперь ретранслировали почту извне организации.

На практике данный протокол не получил такую популярность среди вредоносных программ как HTTP. Но несмотря на это некоторые приложения, например Sality или ProjectSauron, используют именно его для обмена информацией с управляющим сервером.