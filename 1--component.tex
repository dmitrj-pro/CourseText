\section{Компоненты DLP}

Выделяют три основных компонента DLP систем:

\begin{itemize}
	\item
	    Network DLP
	\item 
	    Endpoint DLP
	\item 
	    Storage DLP
	    
\end{itemize}

\textbf{Network DLP}

Данная система, как правило, представляет собой аппаратное решение или программное обеспечение, которое устанавливается в точках сети, исходящих вблизи периметра. Такая система анализирует сетевой трафик по любому каналу TCP/IP или UDP. Система контролирует обмен информацией через IM и пиринговые системы, а также позволяет блокировать пересылку информации через HTTP(s), FTP(s) и SMTP-каналы с возможностью информирования пользователей о нарушении корпоративной политики.

\textbf{Endpoint DLP}

Endpoint DLP производят контроль перемещения информации на устройствах. Позволяет в режиме реального времени отслеживать, а в случае обнаружения запрещенного контента — и блокировать попытки копирования информации на съемные носители информации, печать и отсылку по факсу, по электронной почте и т. д.

Системы такого типа функционируют на рабочих станциях конечных пользователей или серверах в организациях. Конечная точка, как и в других сетевых системах, может быть обращена как к внутренним, так и к внешним связям и, следовательно, может использоваться для контроля потока информации между типами и группами пользователей. Прежде, чем данные сообщения будут загружены на устройство, они проверяются сервисом. При содержании неблагоприятного запроса сообщения будут заблокированы. Таким образом они становятся неоправленными и не попадают под действие правил хранения информации на устройстве.

Преимущество DLP системы заключается в том, что она может контролировать и управлять доступом к устройствам физического типа, а также получать доступ к информации до того, как она будет зашифрована. Некоторые системы, которые функционируют на основе конечных утечек, могут также обеспечить контроль приложений с целью блокировки попыток передачи конфиденциальной информации и обеспечения незамедлительной обратной связи с пользователем. Недостаток таких систем заключается в том, что они должны быть установлены на каждом устройстве и не могут использоваться на мобильных устройствах. 
\textbf{Storage DLP}

Storage DLP производит контроль соблюдения процедур хранения конфиденциальных данных. Позволяет в режиме сканирования обнаруживать хранимые конфиденциальные данные на файловых, почтовых, Web-серверах, в системах документооборота, на серверах баз данных и т. д. Проводит анализ легитимности хранения данных в их текущем местонахождении и перенос при необходимости в защищенные хранилища.