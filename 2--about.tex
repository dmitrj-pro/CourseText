\section{Что такое утечка информации}

Утечку информации  в общем плане можно рассматривать как неправомерный выход конфиденциальных сведений за пределы организации или круга лиц, которым эти сведения были доверены.

Утечка информации по своей сущности всегда предполагает противоправное (тайное или явное, осознанное или случайное) овладение конфиденциальной информацией, независимо от того, каким путем это достигается.

Утечку охраняемой информации, может произойти при наличии ряда обстоятельств. Если есть злоумышленник, который такой информацией интересуется и затрачивает определенные силы и средства для ее получения. И если есть условия, при которых он может рассчитывать на овладение интересующую его информацию (затратив на это меньше сил, чем если бы он добывал ее сам).

Что касается причин и условий утечки информации, то они, при всех своих различиях, имеют много общего.

Причины связаны, как правило, с несовершенством норм по хранению секретной информации, а также с нарушением этих норм (в том числе и несовершенных), отступлением от правил обращения с соответствующими документами, техническими средствами, образцами продукции и других материалов, содержащих конфиденциальную информацию.
